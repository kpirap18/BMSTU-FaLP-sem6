\section*{Практические вопросы}


<<<<<<< Updated upstream
\subsubsection*{1. Написать функцию, которая по своему списку-аргументу lst определяет является ли он палиндромом (то есть равны ли lst и (reverse lst)).}
=======
%\subsubsection*{1.}
>>>>>>> Stashed changes
\begin{lstlisting}[language=Lisp]
(defun palindrome-p (list)
	(let* ((length (length list))
		(half-length (ceiling length 2))
		(tail (nthcdr half-length list))
		(reversed-head (nreverse (butlast list half-length))))
		(equal tail reversed-head)))
<<<<<<< Updated upstream

(defun is-palindrome-2 (lst)
	(equalp lst (reverse lst)))
\end{lstlisting}


\subsubsection*{2. Написать предикат set-equal, который возвращает t, если два его множествааргумента содержат одни и те же элементы, порядок которых не имеет значения}

=======
\end{lstlisting}

%\subsubsection*{2.}
>>>>>>> Stashed changes
\begin{lstlisting}[language=Lisp]
(defun my-subsetp (set1 set2)
	(reduce
		#'(lambda (acc1 set1-el)
			(and acc1 (reduce
				#'(lambda (acc2 set2-el)
<<<<<<< Updated upstream
					(or acc2 (= set2-el set1-el))) 
				set2 :initial-value Nil)))
	set1 :initial-value T))

(defun set-equal (set1 set2)
	(if (= (length set1) (length set2))
		(and (my-subsetp set1 set2) (my-subsetp set2 set1))
		Nil))
\end{lstlisting}

\newpage
\subsubsection*{3. Напишите свои необходимые функции, которые обрабатывают таблицу из 4-х точечных пар: (страна . столица), и возвращают по стране - столицу, а по столице — страну .}
\begin{lstlisting}[language=Lisp]
(defun my-assoc (key table)
	(cond ((null table) nil)
		((equal key (caar table)) (cdar table))
		(T (my-assoc key (cdr table))))))
=======
					(or acc2 (= set2-el set1-el))) set2 :initial-value Nil)))
	set1 :initial-value T))

(defun set-equal (set1 set2)
	(cond ((= (length set1) (length set2))
			(and (my-subsetp set1 set2) (my-subsetp set2 set1)))
		(T Nil)))
\end{lstlisting}


%\subsubsection*{3.}
\begin{lstlisting}[language=Lisp]
; столицу
(defun my-assoc (key table)
	(cond ((null table) nil)
		((equal val (caar table)) (cdar table))
		(T (my-rassoc val (cdr table)))))
>>>>>>> Stashed changes

; страну
(defun my-rassoc (val table)
	(cond ((null table) nil)
		((equal val (cdar table)) (caar table))
		(T (my-rassoc val (cdr table)))))

; с помощью стандартных функций
(defun find-capital (key table)
	(cdr (assoc key table)))

(defun find-country (key table)
	(car (rassoc key table)))
<<<<<<< Updated upstream
\end{lstlisting}

\subsubsection*{4. Напишите функцию swap-first-last, которая переставляет в списке-аргументе первый и последний элементы.}
=======

\end{lstlisting}


%\subsubsection*{4.}
>>>>>>> Stashed changes
\begin{lstlisting}[language=Lisp]
(defun f1 (lst)
	(reverse (cdr (reverse lst))) )

(defun swap-first-last (lst)
<<<<<<< Updated upstream
	(append (append (last lst) (cdr (f1 lst))) 
			(cons (car lst) Nil)) )

(defun swap-first-last2 (lst)
	(append (append (last lst) (butlast (cdr lst) 1)) 
			(cons (car lst) Nil))) )

\end{lstlisting}

\newpage
\subsubsection*{5. Напишите функцию swap-two-ellement, которая переставляет в списке- аргументе два указанных своими порядковыми номерами элемента в этом списке.}
\begin{lstlisting}[language=Lisp]
(defun cut-list (lst n l) 
	(cond ((zerop l) nil)
		((and (= n 1) (> l 0)) (cons (car lst) 
				(cut-list (cdr lst) 1 (- l 1))))
		(t (cut-list (cdr lst) (- n 1) l))))

(defun task (lst p q)
	(cond ((= p q) lst)
		((> p q) (task lst q p))
		(t  (let* ((ls (length lst))
					(l (cut-list lst 1 (- p 1)))
					(m (cut-list lst p (- q p)))
					(r (cut-list lst q (- ls q -1))))
			(append l (list (car r)) (cdr m) (list (car m)) (cdr r))))))

\end{lstlisting}

\subsubsection*{6. Напишите две функции, swap-to-left и swap-to-right, которые производят одну круговую перестановку в списке-аргументе влево и вправо, соответственно.}
=======
	(append (append (last lst) (cdr (f1 lst))) (cons (car lst) Nil)) )

(defun swap-first-last2 (lst)
	(append (append (last lst) (butlast (cdr lst) 1)) (cons (car lst) Nil))) )

\end{lstlisting}



%\subsubsection*{6.}
>>>>>>> Stashed changes
\begin{lstlisting}[language=Lisp]
(defun rot-left(n l)
	(append (nthcdr n l) (butlast l (- (length l) n))))

(defun rot-right(n l)
	(rot-left (- (length l) n) l))
<<<<<<< Updated upstream
\end{lstlisting}


\newpage
\subsubsection*{9. Напишите функцию, select-between, которая из списка-аргумента из 5 чисел выбирает	только те, которые расположены между двумя указанными границами-аргументами и	возвращает их в виде списка (упорядоченного по возрастанию списка чисел (+ 2 балла)).}
\begin{lstlisting}[language=Lisp]
(defun find-elements (lst left right)
	(remove-if #'(lambda (x) (null x))
		(mapcar #'(lambda (x) (cond (< left x right) x)) lst)))

(defun find-min (lst)
	(setf temp (car lst))
	(mapcar #'(lambda (x)
		(cond (> temp x) (setf temp x))) lst) temp)

(defun set-element (new old lst)
	(cond ((= (car lst) old) (rplaca lst new))
		(t (set-element new old (cdr lst)))) lst)

(defun my-sort (lst)
	(maplist #'(lambda (x)
		(and (setf temp (find-min x))
			(set-element (car x) temp x)
			(rplaca x temp))) lst) lst)

(defun select-between (lst b1 b2)
	(cond ((null lst) nil)
		((not (and (numberp b1) (numberp b2))) nil)
		((= b1 b2) nil)
		((> b1 b2)(my-sort (find-elements lst b2 b1)))
		((> b2 b1)(my-sort (find-elements lst b1 b2)))))
\end{lstlisting}
=======
\end{lstlisting}
>>>>>>> Stashed changes
