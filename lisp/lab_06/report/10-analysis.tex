\section*{Практические вопросы}


\subsubsection*{1. Напишите функцию, которая уменьшает на 10 все числа из списка-аргумента этой функции.}
\begin{lstlisting}[language=Lisp]
; одноуровневые список
; только числа
(defun minus-d (lst)
	(mapcar #'(lambda (x) (- x 10)) lst))

; смешанный список
(defun minus-d (lst)
	(mapcar #'(lambda (x) (cond ((numberp x) (- x 10)) (x))) lst))

; рекурсия
; только числа
(defun f (lst res)
	(cond ((null lst) (reverse res))
		(T (f (cdr lst) (cons (- (car lst) 10) res))) ) )

; смешанный список
(defun f (lst res)
	(cond ((null lst) (reverse res))
		((numberp (car lst)) (f (cdr lst) (cons (- (car lst) 10) res)))
		(T (f (cdr lst) (cons (car lst) res))) ) )

(defun my-minus (lst)
	(f lst ()))

; структурированный список
; вспомогательная функция
(defun dop-fun (lst)
	(cond ((and (numberp (car lst)) (numberp (cdr lst))) 
			(cons (- (car lst) 10) (- (cdr lst) 10)))
		((and (symbolp (car lst)) (numberp (cdr lst))) 
			(cons (car lst) (- (cdr lst) 10)))
		((and (numberp (car lst)) (symbolp (cdr lst))) 
			(cons (- (car lst) 10) (cdr lst)))
		((and (atom (cdar lst)) (numberp (cdr lst))) 
			(cons (dop-fun (car lst)) (- (cdr lst) 10)))
		((and (atom (cdar lst)) (symbolp (cdr lst))) 
			(cons (dop-fun (car lst)) (cdr lst)))
		(T lst)))

; функционал
(defun minus-d-all (lst)
	(mapcar #'(lambda (x) 
		(cond ((numberp x) (- x 10))
			((atom x) x)
			((atom (cdr x)) (dop-fun x))
			((listp x) (minus-d-all x)))) lst))




; рекурсия
(defun f (lst)
	(cond ((null lst) ())
		((symbolp (car lst)) (cons (car lst) (f (cdr lst))))
		((numberp (car lst)) (cons (- (car lst) 10) (f (cdr lst))))
		((atom (car lst)) (cons (car lst) (f (cdr lst) )))
		((atom (cdr lst)) (cons (dop-fun (car lst)) (f (cdr lst) )))
		(T (cons (f (car lst)) (f (cdr lst))))) )

; "хвостовая" рекурсия
(defun f (lst res)
	(cond ((null lst) (reverse res))
		((symbolp (car lst)) (f (cdr lst) (cons (car lst) res)))
		((numberp (car lst)) (f (cdr lst) (cons (- (car lst) 10) res)))
		((atom (car lst)) (f (cdr lst) (cons (car lst) res)))
		((atom (cdr lst)) (f (cdr lst) (cons (dop-fun (car lst)) res)))
		(T (f (cdr lst) (cons (f (csr lst) ()) res))) ) )

\end{lstlisting}

\newpage
\subsubsection*{2. Напишите функцию, которая умножает на заданное число-аргумент все числа из заданного списка-аргумента, когда \newline a) все элементы списка --- числа,\newline б) элементы списка -- любые объекты}

\begin{lstlisting}[language=Lisp]
; одноуровневые список
; только числа
(defun mult-f (lst num)
	(mapcar #'(lambda (x) (* x num)) lst))

; смешанный список
(defun mult-f (lst num)
	(mapcar #'(lambda (x) (cond ((numberp x) (* x num)) (x))) lst))

; рекурсия
; только числа
(defun f (lst num res)
	(cond ((null lst) (reverse res))
		(T (f (cdr lst) (cons (* (car lst) num) res))) ) )

; смешанный список
(defun f (lst num res)
	(cond ((null lst) (reverse res))
		((numberp (car lst)) (f (cdr lst) (cons (* (car lst) num) res)))
		(T (f (cdr lst) (cons (car lst) res))) ) )

(defun my-mult (lst num)
	(f lst num ()))

; структурированный список
; вспомогательная функция
(defun dop-fun (lst num)
	(cond ((and (numberp (car lst)) (numberp (cdr lst))) 
		(cons (* (car lst) num) (* (cdr lst) num)))
	((and (symbolp (car lst)) (numberp (cdr lst))) 
		(cons (car lst) (* (cdr lst) num)))
	((and (numberp (car lst)) (symbolp (cdr lst)))
		(cons (* (car lst) num) (cdr lst)))
	((and (atom (cdar lst)) (numberp (cdr lst))) 
		(cons (dop-fun (car lst)) (* (cdr lst) num)))
	((and (atom (cdar lst)) (symbolp (cdr lst))) 
		(cons (dop-fun (car lst)) (cdr lst)))
	(T lst)))

; функционал
(defun mult-d-all (lst num)
	(mapcar #'(lambda (x) 
		(cond ((numberp x) (* x num))
			((atom x) x)
			((atom (cdr x)) (dop-fun x))
			((listp x) (minus-d-all x)))) lst))


	
	
	
	
	
	
; рекурсия
(defun f (lst)
	(cond ((null lst) ())
		((symbolp (car lst)) (cons (car lst) (f (cdr lst))))
		((numberp (car lst)) (cons (* (car lst) num) (f (cdr lst))))
		((atom (car lst)) (cons (car lst) (f (cdr lst) )))
		((atom (cdr lst)) (cons (dop-fun (car lst)) (f (cdr lst) )))
		(T (cons (f (car lst)) (f (cdr lst))))) )

; "хвостовая" рекурсия
(defun f (lst res)
	(cond ((null lst) (reverse res))
		((symbolp (car lst)) (f (cdr lst) (cons (car lst) res)))
		((numberp (car lst)) (f (cdr lst) (cons (* (car lst) num) res)))
		((atom (car lst)) (f (cdr lst) (cons (car lst) res)))
		((atom (cdr lst)) (f (cdr lst) (cons (dop-fun (car lst)) res)))
		(T (f (cdr lst) (cons (f (csr lst) ()) res))) ) )

\end{lstlisting}

\newpage
\subsubsection*{3. Написать функцию, которая по своему списку-аргументу lst определяет является ли он палиндромом (то есть равны ли lst и (reverse lst)).}
\begin{lstlisting}[language=Lisp]
(defun palindrome-p (list)
	(let* ((length (length list))
		(half-length (ceiling length 2))
		(tail (nthcdr half-length list))
		(reversed-head (nreverse (butlast list half-length))))
		(equal tail reversed-head)))

(defun is-palindrome-2 (lst)
	(equalp lst (reverse lst)))
\end{lstlisting}

\subsubsection*{4. Написать предикат set-equal, который возвращает t, если два его множества-аргумента содержат одни и те же элементы, порядок которых не имеет значения.}
\begin{lstlisting}[language=Lisp]
; функционал 
(defun my-subsetp (set1 set2)
	(reduce
		#'(lambda (acc1 set1-el)
			(and acc1 (reduce
				#'(lambda (acc2 set2-el)
					(or acc2 (= set2-el set1-el))) set2 :initial-value Nil)))
		set1 :initial-value T))

(defun set-equal (set1 set2)
	(if (= (length set1) (length set2))
		(and (my-subsetp set1 set2) (my-subsetp set2 set1))
		Nil))

; рекурсия 
(defun find-elem-in-set (set1 elem) 
	(cond ((null set1) Nil)
		((= (car set1) elem) T)
		(T (find-elem-in-set (cdr set1) elem)) ) )

(defun set-equal-rec (set1 set2) 
	(cond ((null set1))
		((find-elem-in-set set2 (car set1)) (set-equal-rec (cdr set1) set2))
		(T Nil)) )

(defun set-equal (set1 set2)
	(if (= (length set1) (length set2)) 
		(set-equal-rec set1 set2) Nil) )
\end{lstlisting}

\newpage
\subsubsection*{5. Написать функцию которая получает как аргумент список чисел, а возвращает список квадратов этих чисел в том же порядке.}
\begin{lstlisting}[language=Lisp]
; одноуровневые список
; только числа
(defun mult-f (lst)
	(mapcar #'(lambda (x) (* x x)) lst))

; смешанный список
(defun mult-f (lst)
	(mapcar #'(lambda (x) (cond ((numberp x) (* x x)) (x))) lst))

; рекурсия
; только числа
(defun f (lst res)
	(cond ((null lst) (reverse res))
		(T (f (cdr lst) (cons (* (car lst) (car lst)) res))) ) )

; смешанный список
(defun f (lst res)
	(cond ((null lst) (reverse res))
		((numberp (car lst)) (f (cdr lst) (cons (* (car lst) (car lst)) res)))
		(T (f (cdr lst) (cons (car lst) res))) ) )

(defun my-mult (lst num)
	(f lst num ()))

; структурированный список
; вспомогательная функция
(defun dop-fun (lst)
	(cond ((and (numberp (car lst)) (numberp (cdr lst))) 
		(cons (* (car lst) (car lst)) (* (cdr lst) (car lst))))
	((and (symbolp (car lst)) (numberp (cdr lst))) 
		(cons (car lst) (* (cdr lst) (car lst))))
	((and (numberp (car lst)) (symbolp (cdr lst))) 
		(cons (* (car lst) (car lst)) (cdr lst)))
	((and (atom (cdar lst)) (numberp (cdr lst))) 
		(cons (dop-fun (car lst)) (* (cdr lst) (car lst))))
	((and (atom (cdar lst)) (symbolp (cdr lst))) 
		(cons (dop-fun (car lst)) (cdr lst)))
	(T lst)))

; функционал
(defun mult-d-all (lst)
	(mapcar #'(lambda (x) 
		(cond ((numberp x) (* x x))
			((atom x) x)
			((atom (cdr x)) (dop-fun x))
			((listp x) (minus-d-all x)))) lst))


; рекурсия
(defun f (lst)
	(cond ((null lst) ())
		((symbolp (car lst)) (cons (car lst) (f (cdr lst))))
		((numberp (car lst)) (cons (* (car lst) (car lst)) (f (cdr lst))))
		((atom (car lst)) (cons (car lst) (f (cdr lst) )))
		((atom (cdr lst)) (cons (dop-fun (car lst)) (f (cdr lst) )))
		(T (cons (f (car lst)) (f (cdr lst))))) )

; "хвостовая" рекурсия
(defun f (lst res)
	(cond ((null lst) (reverse res))
		((symbolp (car lst)) (f (cdr lst) (cons (car lst) res)))
		((numberp (car lst)) (f (cdr lst) (cons (* (car lst) (car lst)) res)))
		((atom (car lst)) (f (cdr lst) (cons (car lst) res)))
		((atom (cdr lst)) (f (cdr lst) (cons (dop-fun (car lst)) res)))
		(T (f (cdr lst) (cons (f (csr lst) ()) res))) ) )

\end{lstlisting}

\newpage
\subsubsection*{6. Напишите функцию, select-between, которая из списка-аргумента, содержащего только числа, выбирает только те, которые расположены между двумя указанными границами-аргументами и возвращает их в виде списка (упорядоченного по возрастанию списка чисел (+ 2 балла)).}
\begin{lstlisting}[language=Lisp]
(defun bubble_move (lst)
	(cond ((atom (cdr lst)) lst)
		((> (car lst) (cadr lst)) 
			(cons (cadr lst) (bubble_move (cons (car lst) (cddr lst)))))
		(T lst) ))

(defun my-sort (lst)
	(cond ((atom (cdr lst)) lst)
		(T (bubble_move (cons (car lst) (my-sort (cdr lst)))))))

;--------------------
(defun find-elements (lst left right)
	(remove-if #'(lambda (x) (null x))
		(mapcar #'(lambda (x) (if (< left x right) x)) lst)))

(defun select-between (lst b1 b2)
	(cond ((null lst) nil)
		((not (and (numberp b1) (numberp b2))) nil)
		((= b1 b2)(and (print "ERROR: wrong format of borders (equal)") nil))
		((> b1 b2)(my-sort (find-elements lst b2 b1)))
		((> b2 b1)(my-sort (find-elements lst b1 b2)))))
;--------------------
; Рекурсивно. Для смешанного списка.
(defun select-rec-one-lvl (lst a b res)
	(cond ((null lst) res)
		((and 
			(numberp (car lst)) 
			(<= (car lst) b) 
			(>= (car lst) a)) 
			(select-rec-one-lvl (cdr lst) a b (cons (car lst) res)))
		(t (select-rec-one-lvl (cdr lst) a b res))))
(defun select-between (lst)
	(select-rec-one-lvl lst 1 10 ()))

; также номер 8 из ЛР 7
\end{lstlisting}

\newpage
\subsubsection*{7. Написать функцию, вычисляющую декартово произведение двух своих списков-аргументов. (Напомним, что А х В это множество всевозможных пар (a b), где а принадлежит А, принадлежит В.)}
\begin{lstlisting}[language=Lisp]
; функционалы
(defun decart (lstx lsty)
	(mapcan #'(lambda (x)
		(mapcar #'(lambda (y)
		(list x y)) lsty)) lstx))

; рекурсия без накапливаний cons
(defun decart-rec (el lst2 res)
	(cond ((null lst2) res)
	(t (decart-rec el (cdr lst2) (cons (cons el (cons (car lst2) Nil)) res) ))))

(defun decart (lst1 lst2 res)
	(cond ((null lst1) res)
	(t (decart (cdr lst1) lst2 (decart-rec (car lst1) lst2 res)))))

; просто рекурсия 
(defun decart-elem (lst elem)
	(cond ((null lst) ())
	(T (cons (list elem (car lst)) (decart-elem (cdr lst) elem)))) )

(defun decart (lst1 lst2)
	(cond ((null lst1) nil)
	(T (append (decart-elem lst2 (car lst1)) (decart (cdr lst1) lst2)))) )

\end{lstlisting}

\newpage
\subsubsection*{8. Почему так реализовано reduce, в чем причина? \newline (reduce \#\' + ()) -> 0}
У reduce есть особый параметр \begin{verbatim}initial-value \end{verbatim} который помещается перед последовательностью и затем применяется функция. Его можно задать, используя:
\begin{verbatim}
(reduce #’+ () :initial-value 100)  -> 100
\end{verbatim}

По умолчанию, для сложения оно равно 0, для умножения 1.


Случай ниже выдаст ошибку
\begin{verbatim}
 (reduce \#\' + 0) -> 0 
\end{verbatim}
Ошибка:
\begin{verbatim}
debugger invoked on a TYPE-ERROR in thread
The value
0
is not of type
SEQUENCE
\end{verbatim}

\newpage
\subsubsection*{9. Пусть list-of-list список, состоящий из списков. Написать функцию, которая вычисляет сумму длин всех элементов list-of-list, т.е. например для аргумента ((1 2) (3 4)) -> 4.}
\begin{lstlisting}[language=Lisp]
; функционал
(defun list-of-list (lst)
	(reduce #'(lambda (acc lst) (+ acc (length lst)))
		(cons 0 lst)))

(defun my-length-rec (lst n)
	(cond 
		((null lst) n)
		(T (my-length-rec (cdr lst) (+ n 1)))) )	

(defun my-length (lst)	
	(my-length-rec lst 0) )

(defun list-of-list-rec (lst len)
	(cond ((null lst) len)
		((atom (car lst)) (list-of-list-rec (cdr lst) (+ len 1)))
		((and (atom (caar lst)) (atom (cdar lst))) (list-of-list-rec (cdr lst) (+ len 2)))
		(T (list-of-list-rec (cdr lst) (+ len (my-length (car lst)) )))))

(defun list-of-list (lst)
	(list-of-list-rec lst 0))
\end{lstlisting}
