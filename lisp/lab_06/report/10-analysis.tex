\section*{Практические вопросы}


\subsubsection*{1. Напишите функцию, которая уменьшает на 10 все числа из списка-аргумента этой функции.}

\subsubsection*{2. Напишите функцию, которая умножает на заданное число-аргумент все числа из заданного списка-аргумента, когда \newline a) все элементы списка --- числа,\newline б) элементы списка -- любые объекты}


\subsubsection*{3. Написать функцию, которая по своему списку-аргументу lst определяет является ли он палиндромом (то есть равны ли lst и (reverse lst)).}


\subsubsection*{4. Написать предикат set-equal, который возвращает t, если два его множества-аргумента содержат одни и те же элементы, порядок которых не имеет значения.}


\subsubsection*{5. Написать функцию которая получает как аргумент список чисел, а возвращает список квадратов этих чисел в том же порядке.
}


\subsubsection*{6. Напишите функцию, select-between, которая из списка-аргумента, содержащего только числа, выбирает только те, которые расположены между двумя указанными границамиаргументами и возвращает их в виде списка (упорядоченного по возрастанию списка чисел (+ 2 балла)).
}


\subsubsection*{7. Написать функцию, вычисляющую декартово произведение двух своих списковаргументов. (Напомним, что А х В это множество всевозможных пар (a b), где а принадлежит А, принадлежит В.)}


\subsubsection*{8. Почему так реализовано reduce, в чем причина? \newline (reduce \#\' + 0) -> 0 \newline (reduce \#\' + ()) -> 0}

\subsubsection*{9. Пусть list-of-list список, состоящий из списков. Написать функцию, которая вычисляет сумму длин всех элементов list-of-list, т.е. например для аргумента ((1 2) (3 4)) -> 4.}

